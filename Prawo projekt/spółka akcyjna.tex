\documentclass[a4paper, 11pt]{article}
\usepackage[polish]{babel}
\usepackage[MeX]{polski}
\usepackage[utf8]{inputenc}
\usepackage[T1]{fontenc}
%\usepackage{times}
\usepackage{graphicx,wrapfig}
%\usepackage{anysize}
%\usepackage{tikz}
%\usetikzlibrary{calc,through,backgrounds,positioning}
\usepackage{anysize}
\usepackage{float}
%\usepackage{stmaryrd}
%\usepackage{amssymb}
%\usepackage{amsthm}
%\marginsize{3cm}{3cm}{3cm}{3cm}
%\usepackage{amsmath}
%\usepackage{color}
%\usepackage{listings}
%\usepackage{enumerate}
%\lstloadlanguages{Ada,C++}


\begin{document}	
	% \noindent -  w tym akapicie nie bedzie wciecia
	% \ indent - to jest aut., ale powoduje ze jest wciecie
	% \begin{flushleft}, flushright, center - wyrownianie akapitu
	% \textbf{pogrubiany tekst}
	% \textit{kursywa} 
	% 					STRONY 
	%  http://www.codecogs.com/latex/eqneditor.php 
	%  http://www.matematyka.pl/latex.htm
	% 
	\begin{titlepage}
	
	
		
		\newcommand{\HRule}{\rule{\linewidth}{0.5mm}} % Defines a new command for the horizontal lines, change thickness here
		
		\center % Center everything on the page
		
		%----------------------------------------------------------------------------------------
		%	HEADING SECTIONS
		%----------------------------------------------------------------------------------------
		
		\textsc{\LARGE Akademia Górniczo-Hutnicza im. Stanisława Staszica}\\[1.5cm] % Name of your university/college
		\textsc{\Large Kraków}\\[0.5cm] % Major heading such as course name
		\textsc{\large }\\[0.5cm] % Minor heading such as course title
		
		%----------------------------------------------------------------------------------------
		%	TITLE SECTION
		%----------------------------------------------------------------------------------------
		
		\HRule \\[0.4cm]
		{\fontsize{38}{50}\selectfont Spółka akcyjna}
	%	{ \Huge \bfseries} Symulator pożaru lasu\\[0.3cm] % Title of your document
		\HRule \\[1.5cm]
		
		%----------------------------------------------------------------------------------------
		%	AUTHOR SECTION
		%----------------------------------------------------------------------------------------
		
		% If you don't want a supervisor, uncomment the two lines below and remove the section above
		\Large \emph{Autorzy:}\\
		Marcin \textsc{Jędrzejczyk}\\ % Your name
		Sebastian \textsc{Katszer}\\ % Your name
		Katarzyna \textsc{Kosiak} \\ % Your name
		Paweł	  \textsc{Ogorzały} \\ [3cm]\

		
		%----------------------------------------------------------------------------------------
		%	DATE SECTION
		%----------------------------------------------------------------------------------------
		
		{\large \today}\\[3cm] % Date, change the \today to a set date if you want to be precise
		
		%----------------------------------------------------------------------------------------
		%	LOGO SECTION
		%----------------------------------------------------------------------------------------
		
		%\includegraphics{Logo}\\[1cm] % Include a department/university logo - this will require the graphicx package
		
		%----------------------------------------------------------------------------------------
		
		\vfill % Fill the rest of the page with whitespace
		
	\end{titlepage}
	
	%SPIS TRESI
	%
	%
	%
	%
	%
	%
	%
	%
	%
	
	\tableofcontents
	\vfill

	\newpage
	\section{Informacje wstępne}

	\subsection{Cel powstania firmy}
	\indent

Celem powstania firmy Przedsiębiorstwa wytwórczo-handlowego "Platri" była chęć stworzenia organizacji, która by produkowała i sprzedała gry planszowe, karciane, łamigłówki, puzzle, wytworzone w bardzo wysokim standardzie. Innymi czynnikami wpływającymi na wcielenie tego przedsięwzięcia w życie jest między innymi rozszerzenie grona graczy, produkcja nowych gier, rozsławienie twórców jak i emisja gier rangi kolekcjonerskiej. Mamy nadzieję, że "Platri" odniesie sukces i pomoże rozwinąć w młodych ludziach zdolność analitycznego myślenia, w przeciwieństwie do oglądania telewizji, a zgromadzony przychód tylko przyśpieszy ten rozwój.

	\subsection{Okoliczności} 
	\indent
		
	Na polskim rynku istnieje tylko kilka firm produkcyjnych, które oferują gry poważne dla zaawansowanego odbiorcy, takich jak: REBEL.pl, Galakta, Poratal. Coraz większe zainteresowanie wśród osób w przedziale wiekowym 16-40, jest zainteresowane grami planszowymi. W związku z tym istnieje zapotrzebowanie na takie produkty, a więc chcemy wyjść na przeciw odbiorcom i udostępnić im rozrywkę najwyższej jakości.


	
	\subsection{Działalność przedsiębiorstwa}
	
	\begin{description}
	\item[Obszar:]	Polska
	\item[Branża:]	Zabawki
	\item[Rynek:]	Gry planszowych, karciane, łamigłówki, puzzle
	\item[Docelowa grupa klientów:]	Osoby w wieku >6 lat, głównie przedział wiekowy 20-40
	\item[Przedmiot działalności:]	Działalność wytwórcza i handlowa		
	\end{description}


	\subsection{Forma prawna przedsiębiorstwa}
%(Ctrl + c, Ctrl + v -- http://biznes.pl/firma/zakladam-firme/spolka-akcyjna/whblfd, http://mfiles.pl/pl/index.php/Spółka_akcyjna,)

	\subsubsection{Opis}
	\indent	

	Spółka akcyjna jest spółką kapitałową, uregulowaną w Kodeksie spółek handlowych. Jest to forma działalności, której celem jest zebranie kapitału od bardzo wielu akcjonariuszy (udziałowców).  Kapitał akcyjny dzieli się na akcje o równej wartości nominalnej, zaliczane są one do papierów wartościowych. Akcje są zbywalne i mogą być dopuszczone do obrotu na giełdzie. Akcja stanowi podstawę nabycia praw wspólnika (akcjonariusza), może mieć charakter imienny lub na okaziciela, zwykły lub uprzywilejowany co do prawa głosu, dywidendy bądź podziału majątku w razie likwidacji spółki akcyjnej. Jest formą działalności przeznaczoną dla prowadzenia średnich i dużych przedsiębiorstw. 

 \subsubsection{Cechy}%	(http://mfiles.pl/pl/index.php/Spółka_akcyjna)
\begin{itemize}
\item Umowa spółki z akcyjnej wymaga formy aktu notarialnego,
\item Akcjonariusze nie odpowiadają za zobowiązania spółki,
\item Spółka posiada osobowość prawną,
\item Kapitał zakładowy nie mniejszy niż 100.000 zł,
\item Kluczowe znaczenie w SA mają akcje, czyli udziały w spółce, o równej wartości nominalnej (wartość nominalna akcji nie może być niższa niż 1 grosz),
\item Akcje mogą się dzielić na zwykle i uprzywilejowane (pod względem liczby głosów na walnym zgromadzeniu akcjonariuszy czy prawa do dywidendy),
\item Założenie spółki akcyjnej może nastąpić przez samych założycieli (ewentualnie łącznie z osobami trzecimi) lub w drodze publicznej subskrypcji (ogłoszeń o zapisach na akcje),
\item Spółka jest wpisana do Krajowego Rejestru Sadowego.
\end{itemize}








	\subsection{przeprowadzone postępowanie przygotowawcze}
%(http://docplayer.pl/3048046-Postepowanie-przygotowawcze-do-zalozenia-firmy.html)
	\indent
	

	Etap ten obejmuje inkubację pomysłu na przedsiębiorczość poprzez sformowanie odpowiednich celów przedsięwzięcia oraz określenie alternatyw. Składa się on z:

	\subsubsection{Wykreowanie pomysłu}
	\indent
	
	Nasz pomysł zrodził się podczas tworzenia gry komputerowej na bazie gry planszowej “Osadnicy z Catanu”. Zrozumieliśmy wtedy, że wykreowanie gry rodzinnej, nie jest bardzo skomplikowanym procesem. Niemniej jednak jest on długi, a po jego ukończeniu, twórcy muszą zwrócić się do producenta. Chcielibyśmy pomóc osobom w wprowadzeniu swoich pomysłów na rynek w bardzo dobrej jakości wyprodukowania i przepiękną grafiką.

	\subsubsection{Analiza rynku}
	\indent
	
	Na rynku polskim nie istnieje wiele producentów gier planszowych. Wykonane przez nich produkty, nie spełniają oczekiwaniom graczy. Twórcy, niejednokrotnie nie są w stanie znaleźć producenta chętnego na wytworzenie jego tworu intelektualnego. Naszym celem jest wykorzystanie tych braków na naszym rodzimym rynku do rozrostu naszej firmy do skali światowej.

%Opis rynku (sprzedawany towar, konsumenci, konkurencja, przewidywane obroty itd.)
%Potrzeby lokalowe (określenie wymogów m.in. BHP, lokalizacja)
%Wybór formy prawnej
%Rozpoznaliśmy warunki prowadzenia działalności gospodarczej w Polsce jak i całą procedurę rejestracji. Ze względu na wielkość przedsięwzięcia, nasz wybór padł na spółkę akcyjną.
%Rozważenie innych alternatyw
	
	
	\subsubsection{Opis rynku}
	\indent

	Naszym produktem będą gry planszowe, karciane o bardzo wysokim poziomie wykonania
wraz z odpowiednim zaprojektowaniem wypraski, która to będzie przechowywać elementy gry w sposób zorganizowany z przewidzeniem możliwości ofoliowania kart. Niestety, aktualnie nie istnieje żadna firma, która wdraża taką opcję. Wiele osób jest pedantyczna na punkcie gier i chcą oni dbać o swoje produkty najlepiej jak tylko mogą. My mamy zamiar im w tym pomóc; dlatego też, naszymi konsumentami będą wszyscy aktualni gracze gier planszowych, głównie w przedziale wiekowym 16 > 40 ale i nawet do 99 roku życia. Ci ludzie bardzo szanują swoje gry i są w stanie wydać na nie 100-300 zł za sztukę i wymagają aby produkt był wysokiej jakości. Naszym największą konkurencją jest REBEL.pl, który mimo wszystko nie produkuję oczekiwanych wyprasek. Przewidywany początkowy obrót miesięczny przy założeniu tylko 5 klientów dziennie wynoszą 30.000 zł. Wraz z rozwojem firmy, liczba ta będzie sukcesywnie wzrastać. Po pierwszym roku działalności przewidywany obrót wyniesie 120.000, a już po kolejnym roku 300.000 zł. Przy takich założeniach, obrót roczny wynosi: 3.600.000 zł.

 \subsubsection{Potrzeby lokalowe}

	Nasze produkty, sprzedawać będziemy poprzez wysyłkę internetową jak i poprzez tradycyjną sprzedaż.
Dlatego też, lokal powinien mieć co najmniej $20m^{2}$ dla obsługi klienta jak i $60m^{2}$ na potrzeby biurowe i produkcyjne. Istnieje możliwość rozdzielenia tych dwóch płaszczyzn na 2 różne lokale.
	
\section{Rejestracja firmy}
Rejestracja firmy:
-wymienić w punktach i krótko opisać (w odpowiedniej kolejności) instytucje do których należy się udać, aby móc rozpocząć własną działalność gospodarczą (np.notariusz, urząd gminy, Krajowy Rejestr Sądowy, Urząd Statystyczny, Urząd Skarbowy, bank, ZUS etc.)
\begin{enumerate}

\item NOTARIUSZ - Akt założycielski i statut spółki
%https://docs.google.com/document/d/1ScEQRKwYyo9Ded-5cSrE0V3SFPNR3UDVlBebcDdA9mU/edit#heading=h.z945irdmqjj0

Pierwszym etapem założenia Spółki jest sporządzenie jej statutu. Statut ten powinien być sporządzony w formie aktu notarialnego. Statut musi zawierać::
firmę i siedzibę spółki,
przedmiot jej działalności,
czas trwania jeżeli jest określony,
wysokość kapitału zakładowego oraz kwotę wpłaconą przed zarejestrowaniem na pokrycie kapitału zakładowego,
wartość nominalną akcji i ich liczbę ze wskazaniem, czy akcje są imienne, czy na okaziciela,
liczbę akcji poszczególnych rodzajów i związane z nimi uprawnienia, jeżeli mają być wprowadzone akcji różnych rodzajów,
nazwiska i imiona albo firmy (nazwy) założycieli,
liczbę członków zarządu i rady nadzorczej albo co najmniej minimalną lub maksymalną liczbę członków tych organów oraz podmiot uprawniony do ustalenia składu zarządu lub rady nadzorczej, pismo do ogłoszeń, jeżeli Spółka zamierza dokonywać ogłoszeń również poza Monitorem Sądowym i Gospodarczym

Statut powinien również zawierać, pod rygorem bezskuteczności wobec Spółki, postanowienia dotyczące:
liczby i rodzajów tytułów uczestnictwa w zysku lub w podziale majątku Spółki oraz związanych z nimi praw, wszelkich związanych z akcjami obowiązków świadczenia na rzecz Spółki, poza obowiązkiem wpłacenia należności za akcje,
warunków i sposobu umorzenia akcji,
ograniczeń zbywalności akcji,
co najmniej przybliżoną wielkość wszystkich kosztów poniesionych lub obciążających Spółkę w związku z jej utworzeniem, ustaloną na dzień związania 
Spółki, 
uprawnień osobistych przyznawanych akcjonariuszom, o których mowa w art. 354.


Akt założycielski spółki w formie aktu notarialnego - założyciele wyrażają zgodę na powstanie spółki oraz na treść statutu.
\item Kapitał zakładowy
Kapitał zakładowy spółki akcyjnej jest to wartość majątku spółki określona kwotą pieniężną, jaka jest niezbędna do jej założenia oraz wymagana przez cały okres jej trwania. Kapitał powinien wynosić co najmniej 100 000 zł. Dzieli się go na akcje o równej wartości nominalnej, nie niższej niż 0,01zł. Zgromadzenie wskazanej sumy stanowi niezbędny warunek powstania spółki. Na akcjonariuszach ciąży obowiązek na pokrycie całego kapitału zakładowego.

Art. 329. § 1. Kodeksu spółek handlowych- Akcjonariusz obowiązany jest do wniesienia pełnego wkładu na akcje. § 2. Wpłaty powinny być dokonane równomiernie na wszystkie akcje. Może być on opłacony w formie wkładów pieniężnych i niepieniężnych, jednakże ustawodawca wyłączył możliwość wniesienia jako przedmiotu wkładu praw niezbywalnych, świadczenia pracy bądź usług.

Art. 315. § 1.k.s.h. Wpłaty na akcje powinny być dokonane bezpośrednio lub za pośrednictwem domu maklerskiego, na rachunek spółki w organizacji prowadzony przez bank w Rzeczypospolitej Polskiej.

Dopiero po stwierdzeniu przez sąd, że deklarowane wkłady na pokrycie kapitału zakładowego zostały wniesione, sąd może zarejestrować spółkę, z wyjątkiem przypadków gdy przedmiotem wkładu są wartości niepieniężne, akcje powinny być pokryte aportem najdalej w ciągu roku od dnia zarejestrowania spółki oraz gdy przedmiotem wkładu są pieniądze, akcje powinny być opłacone przed zarejestrowaniem, w co najmniej 25% ich wartości nominalnej ( art. 309 § 3 k.s.h. )

\item Powołanie organów spółki akcyjnej
Kodeks nakłada wymóg powołania zarządu (prowadzenie spraw spółki i jej reprezentowanie), rady nadzorczej(sprawowanie stałego nadzoru nad działalnością spółki) oraz walnego zgromadzenia (najwyższa władza spółki).

\item SĄD - Wpis do rejestru sądowego 
Kolejnym etapem działań jest rejestracja Spółki w Krajowym Rejestrze Sądowym.
 
Zgłoszenie Spółki do rejestru następuje na formularzu rejestracyjnym w terminie nie dłuższym niż sześć miesięcy od jej zawiązania. Jeżeli Spółka nie zostanie zgłoszona do rejestru w ustawowym terminie, zarząd powinien niezwłocznie zawiadomić o tym przez ogłoszenie osoby mające interes prawny oraz zarządzić zwrot wpłaconych sum i wkładów niepieniężnych, a jeśli spółka w organizacji nie jest w stanie dokonać niezwłocznie zwrotu wszystkich wniesionych wkładów lub pokryć w pełni wierzytelności osób trzecich, zarząd dokona likwidacji (jeśli spółka w organizacji nie posiada zarządu, walne zgromadzenie albo sąd rejestrowy ustanawia likwidatora albo likwidatorów).
 
Wniosek o wpis spółki do rejestru (podpisany przez wszystkich członków zarządu) zgłasza zarząd do sądu rejestrowego właściwego ze względu na siedzibę spółki.
 
Z chwilą wpisu do rejestru przedsiębiorców Spółka Akcyjna uzyskuje osobowość prawną.

\item Pozostałe etapy:
zgłoszenie spółki w urzędzie statystycznym,
urzędzie skarbowym,
założenie rachunku bankowego i zgłoszenie spółki w Zakładzie Ubezpieczeń Społecznych.

\end{enumerate}

\subsection{jakie druki należy wypełnić w poszczególnych instytucjach?}
\begin{enumerate}
\item NOTARIUSZ - Akt założycielski i statut spółki:
	
\item Wniesienie kapitału zakładowego 100 000zł na rachunek spółki
\item Powołanie organów spółki akcyjnej
\item SĄD - Wpis do rejestru sądowego: 
Formularz główny\\
KRS W4 	Wniosek o rejestrację podmiotu w rejestrze przedsiębiorców - spółka akcyjna\\
Załączniki:\\
KRS-WA Oddziały, terenowe jednostki organizacyjne\\
KRS-WG 	Emisje akcji\\
KRS-WH Sposób powstania podmiotu\\
KRS-WK Organy podmiotu / wspólnicy uprawnieni do reprezentowania spółki\\
KRS-WL 	Prokurenci, Pełnomocnicy spółdzielni, przedsiębiorstwa państwowego, jednostki badawczo-rozwojowej\\
KRS-WM Przedmiot działalności\\
KRS-ZN Sprawozdania finansowe i inne dokumenty\\

\item URZĄD SKARBOWY:\\
	NIP-2
\item ZUS\\
 ZUS-WPA -wyrejestrowanie
	ZUS-ZPA
\item Urząd Statystyczny\\
	RG-1
\item Bank\\
	Założyć rachunek bankowy
\end{enumerate}



\subsection{jakie dokumenty należy posiadać?}
Do zgłoszenia spółki należy ponadto dołączyć:
\begin{itemize}
\item statut,
\item akty notarialne o zawiązaniu spółki i objęciu akcji,
\item oświadczenie wszystkich członków zarządu, że wymagane statutem wpłaty na akcje oraz wkłady niepieniężne zostały dokonanie zgodnie z prawem,
\item potwierdzony przez bank lub dom maklerski dowód wpłaty na akcje, dokonanej na rachunek spółki w organizacji,
\item dokument stwierdzający ustanowienie organów spółki z wyszczególnieniem ich składu osobowego,
\item zezwolenie lub dowód zatwierdzenia statutu przez właściwy organ władzy publicznej (jeśli są one wymagane do powstania spółki).
\end{itemize}


\subsection{wypełnić konieczne formularze i druki(wzory można ściągnąć z internetu lub udać się do urzędu)}

\section{Formy opodatkowania przedsiębiorstwa}

\subsection{jakie są dopuszczalne formy opodatkowania Państwa przedsiębiorstwa?}
Spółka akcyjna, jako osoba prawna, jest normalnym podatnikiem wszystkich podatków (oczywiście jeżeli wykonuje czynności podlegające opodatkowaniu).
Podlega ona więc opodatkowaniu podatkiem dochodowym (od osób prawnych – CIT), podatkiem od towarów i usług (VAT), podatkiem akcyzowym, podatkiem od czynności cywilnoprawnych, podatkiem od nieruchomości, czy też podatkiem od środków transportowych.


\subsection{jaką formę opodatkowania Państwo wybrali(krótka charakterystyka) i dlaczego ?}

narzucone CIT  za I razem i PIT za II razem od dywident dla wspólników 
W pierwszej kolejności dochód spółki akcyjnej (czyli nadwyżka przychodów nad kosztami) podlega opodatkowaniu na poziomie spółki podatkiem dochodowym od osób prawnych (CIT) wg stawki 19\%. Po opodatkowaniu CIT zysk spółki z o.o. może być wypłacony wspólnikom jako dywidenda. Od wypłaconej dywidendy wspólnicy płacą 19\% PIT. Za pobranie podatku odpowiada spółka akcyjna wspólnicy otrzymują więc wypłatę w kwocie netto (pomniejszoną o podatek).
\section{Zatrudnianie pracowników}
Zatrudnianie pracowników:
\subsection{procedura zatrudniania pracownika (wymagane dokumenty i czynności)}

Umowa o pracę na czas nieokreślony.\\
Umowa o pracę powinna określać:
\begin{itemize}
\item rodzaj pracy
\item miejsce wykonywania pracy
\item wynagrodzenie za pracę odpowiadające rodzajowi pracy, ze wskazaniem składników wynagrodzenia
\item wymiar czasu pracy
\item termin rozpoczęcia pracy.
\end{itemize}

\begin{enumerate}

\item W ciągu 7 dni od dnia zawarcia umowy o pracę pracodawca powinien poinformować każdego pracownika indywidualnie na piśmie o:
obowiązującej dobowej i tygodniowej normie czasu pracy
częstotliwości wypłaty wynagrodzenia za pracę
urlopie wypoczynkowym
długości okresu wypowiedzenia umowy.
\item  Badania lekarskie wstępne
Przed dopuszczeniem pracownika do pracy należy go wysłać na badania lekarskie.
Zgodnie z art. 229 § 4 K. p. pracodawca nie może dopuścić do pracy pracownika bez aktualnego orzeczenia lekarskiego stwierdzającego brak przeciwwskazań do pracy na określonym stanowisku.
Badania lekarskie wykonuje się na koszt pracodawcy i w miarę możliwości w godzinach pracy.
Pracownik zachowuje za ten okres prawo do wynagrodzenia.
\item  Szkolenia BHP
Wg art. 237 kodeksu pracy przed przystąpieniem do pracy i podpisaniem umowy o pracę pracownik musi przejść szkolenie BHP- wstępne i dotyczące stanowiska pracy. Pracownik jest obowiązany potwierdzić na piśmie zapoznanie się z przepisami oraz zasadami bezpieczeństwa i higieny pracy.
Szkolenie pracowników jest przeprowadzane w czasie pracy i na koszt pracodawcy.
\item PIP i sanepid
Wg kodeksu pracy ( art. 209 kp ) w chwili rozpoczęcia prowadzenia działalności gospodarczej pracodawca jest zobowiązany do przekazania właściwemu okręgowemu inspektorowi pracy pisemnego zawiadomienia o miejscu, rodzaju i zakresie prowadzonej działalności. Informację tę należy przekazać do PIP w ciągu 30 dni. Z tym, że początkiem biegu 30 -dniowego terminu będzie zatrudnienie pierwszego pracownika na podstawie umowy o pracę.
Każdy pracodawca, który rozpoczyna swoją działalność, jest też obowiązany w terminie 30 dni od dnia rozpoczęcia tej działalności, czyli od zatrudnienia pierwszego pracownika zawiadomić na piśmie właściwego państwowego inspektora sanitarnego o miejscu, rodzaju i zakresie prowadzonej działalności.
\item Obowiązek wobec ZUS przy zatrudnianiu pracownika
Pracodawca ma obowiązek zgłosić zatrudnioną osobę do ZUS w terminie 7 dni od daty powstania obowiązku ubezpieczenia, czyli od daty zatrudnienia. Obowiązek ten dotyczy zgłoszenia do ubezpieczeń społecznych i ubezpieczenia zdrowotnego osoby zatrudnionej. Zgłoszenia dokonujemy na druku -ZUS ZUA.
\end{enumerate}
Pracodawca powinien zgłosić pracownika do następujących ubezpieczeń:
\begin{itemize}
\item emerytalnego
\item rentowego
\item chorobowego
\item wypadkowego
\item ubezpieczenia zdrowotnego
\end{itemize}

Pracodawca zobowiązany jest także do opłacania składek na Fundusz Pracy oraz Fundusz Gwarantowanych Świadczeń Pracowniczych.

\subsection{zawarcie umowy o dzieło (KUP 50\%), umowy zlecenie, umowy o pracę}
umowa o prace
\section{Wnioski}
 
\subsection{jakie są zalety i wady formy prawnej Państwa przedsiębiorstwa?}
Zalety:\\
Możliwość pozyskiwania kapitału na rynku kapitałowym poprzez emisję akcji czy obligacji 
Dzięki temu spółka może łatwo kumulować i pozyskiwać kapitał od szerokiego kręgu inwestorów.
Brak odpowiedzialności akcjonariuszy za zobowiązania spółki
Dzięki temu, że spółka akcyjna jest osobą prawną, jest odrębnym od swoich akcjonariuszy podmiotem. Oznacza to, że to spółka w swoim imieniu (działając przez swoje organy) zaciąga zobowiązania.
Łatwość kumulacji kapitału
Spółka akcyjna jest nastawiona na duże zyski, które wymagają też dużych inwestycji. Służy temu możliwość emisji przez spółkę instrumentów finansowych, a także prosta procedura pozyskiwania nowych akcjonariuszy (obrót akcjami). 
Możliwość prowadzenia działalności na dużą skalę
Cała struktura spółki akcyjnej podporządkowana jest prowadzeniu nie tylko działalności na dużą skalę, ale też wymagającej wysoce wyspecjalizowanej wiedzy, co wiąże się z nastawieniem na duże zyski.
Łatwe sprawdzenie wiarygodności spółki przez potencjalnych 			kontrahentów i potencjalnych akcjonariuszy
Możliwość założenia spółki akcyjnej przez jedną osobę
Jednoosobowa spółka jest wyjątkiem od reguły, że spółka składa się z co najmniej dwóch wspólników. Wyjątek ten ma uzasadnienie w odniesieniu tylko do spółek kapitałowych, które stanowią połączenie kapitału, a nie indywidualnych cech wspólników. Umożliwia to  przedsiębiorcom, podejmującym działalność na większą skalę, indywidualne prowadzenie własnego przedsiębiorstwa w formie pozwalającej na ograniczenie ryzyka gospodarczego, a także na płacenie niższych podatków.
Możliwość założenia spółki akcyjnej przez obcokrajowców

Wady:\\
Wysoki nominalny kapitał zakładowy
Kapitał zakładowy spółki akcyjnej nie może być niższy niż 100.000 zł. Wysoka wartość kapitału zakładowego spółki akcyjnej może stanowić barierę dla wyboru tej formy prowadzenia działalności.
Kosztowny, skomplikowany i czasochłonny proces rejestracji
Rejestracja wymaga wielu dokumentów, które trzeba złożyć w różnych urzędach i instytucjach, co komplikuje i wydłuża proces rejestracji.
Brak wpływu na działalność spółki przez mniejszych akcjonariuszy
Reprezentacją i prowadzeniem spraw spółki zajmuje się wyspecjalizowany zarząd. Akcjonariusze nie ponoszą, co prawda, żadnej odpowiedzialności za zobowiązania spółki, ale z jej wysokim zyskiem wiążą swój zarobek. 
Podwójne opodatkowanie zysku spółki
Spółka akcyjna opodatkowana jest na dwóch poziomach: akcjonariusza i spółki. Najpierw dochód spółki podlega opodatkowaniu podatkiem dochodowym od osób prawnych (CIT), a w przypadku wypłacenia akcjonariuszom dywidendy – wspólnicy muszą zapłacić podatek dochodowy od osób fizycznych (PIT).
Konieczność występowania rady nadzorczej, niezależnie od ilości 			akcjonariuszy
Zarząd prowadzi sprawy spółki i reprezentuje spółkę. Zarząd składa się tutaj z jednego albo większej liczby członków. Do zarządu mogą być powołane osoby spośród akcjonariuszy lub spoza ich grona. Członków zarządu powołuje i odwołuje rada nadzorcza, chyba że statut spółki stanowi inaczej. Obowiązek istnienia zarządu może być wadą w przypadku, gdy spółka posiada niewielu członków 
Konieczność zatrudniania specjalistycznej obsługi prawnej, 			finansowej i zarządczej
Większość dokumentów w spółce akcyjnej wymaga sporządzenia w formie aktu notarialnego, uchwały podejmowane są w sformalizowanym trybie, obowiązkowe jest prowadzenie pełnej księgowości i corocznego badania sprawozdania finansowego przez biegłego rewidenta. W związku z wysokim stopniem skomplikowania podejmowanych w spółce akcyjnej czynności, a także ze względu na duże rozmiary prowadzonej przez nią działalności, wymaga ona stałego doradztwa zarówno w zakresie finansów, jak i porad prawnych.
Skomplikowany proces likwidacji
Likwidację spółki akcyjnej prowadzą likwidatorzy, podejmując czynności likwidacyjne obejmujące czynności przygotowawcze oraz tzw. właściwe czynności likwidacyjne: zakończenie interesów bieżących spółki, ściągnięcie wierzytelności, upłynnienie majątku spółki, wypełnienie zobowiązań oraz podział pozostałego majątku między akcjonariuszy spółki. W związku z dużą skalą prowadzonej działalności, czynności te wymagają fachowego przeprowadzenia.
Konieczność prowadzenia pełnej księgowości - koszty usług biura rachunkowego lub księgowego
Pełna księgowość to rozbudowany, precyzyjny, skomplikowany i dokładnie sformalizowany system ewidencji zdarzeń gospodarczych. Służy on do kontroli, analizy i generowania informacji na temat sytuacji finansowej firmy w danym momencie lub na przestrzeni jakiegoś okresu.


\subsection{jakie trudności (udogodnienia) napotkali Państwo w trakcie rejestracji firmy?}
Proces rejestracji okazał się być skomplikowany, na szczęście wszystkie potrzebne wzory dokumentów z łatwością dało się znaleźć w Internecie. Niektóre ze znalezionych w Internecie informacji dotyczących zakładania spółki nie miały podanej daty aktualizacji, co utrudniało weryfikację ich poprawności.
\subsection{co Państwa zdaniem przyspieszyłoby proces rejestracji firmy?}
Istnienie strony internetowej, w której na podstawie wyboru rodzaju spółki, wyświetlałby się spis potrzebnych dokumentów - w jednym miejscu, bez szukania po stronach różnych urzędów. 
\subsection{inne nasuwające się wnioski}


\end{document}

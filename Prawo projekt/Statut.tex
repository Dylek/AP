\documentclass[a4paper, 11pt]{article}
\usepackage[polish]{babel}
\usepackage[MeX]{polski}
\usepackage[utf8]{inputenc}
\usepackage[T1]{fontenc}
%\usepackage{times}
\usepackage{graphicx,wrapfig}
%\usepackage{anysize}
%\usepackage{tikz}
%\usetikzlibrary{calc,through,backgrounds,positioning}
\usepackage{anysize}
\usepackage{float}
%\usepackage{stmaryrd}
%\usepackage{amssymb}
%\usepackage{amsthm}
%\marginsize{3cm}{3cm}{3cm}{3cm}
%\usepackage{amsmath}
%\usepackage{color}
%\usepackage{listings}
%\usepackage{enumerate}
%\lstloadlanguages{Ada,C++}


\begin{document}	
	% \noindent -  w tym akapicie nie bedzie wciecia
	% \ indent - to jest aut., ale powoduje ze jest wciecie
	% \begin{flushleft}, flushright, center - wyrownianie akapitu
	% \textbf{pogrubiany tekst}
	% \textit{kursywa} 
	% 					STRONY 
	%  http://www.codecogs.com/latex/eqneditor.php 
	%  http://www.matematyka.pl/latex.htm
	% 
	\begin{titlepage}
	
	
		
		\newcommand{\HRule}{\rule{\linewidth}{0.5mm}} % Defines a new command for the horizontal lines, change thickness here
		
		\center % Center everything on the page
		
		%----------------------------------------------------------------------------------------
		%	HEADING SECTIONS
		%----------------------------------------------------------------------------------------
		
		\textsc{\LARGE Akademia Górniczo-Hutnicza im. Stanisława Staszica}\\[1.5cm] % Name of your university/college
		\textsc{\Large Kraków}\\[0.5cm] % Major heading such as course name
		\textsc{\large }\\[0.5cm] % Minor heading such as course title
		
		%----------------------------------------------------------------------------------------
		%	TITLE SECTION
		%----------------------------------------------------------------------------------------
		
		\HRule \\[0.4cm]
		{\fontsize{38}{50}\selectfont Akt założycielski i statut spółki}
	%	{ \Huge \bfseries} Symulator pożaru lasu\\[0.3cm] % Title of your document
		\HRule \\[1.5cm]
		
		%----------------------------------------------------------------------------------------
		%	AUTHOR SECTION
		%----------------------------------------------------------------------------------------
		
		% If you don't want a supervisor, uncomment the two lines below and remove the section above
		\Large \emph{Autorzy:}\\
		Marcin \textsc{Jędrzejczyk}\\ % Your name
		Sebastian \textsc{Katszer}\\ % Your name
		Katarzyna \textsc{Kosiak} \\ % Your name
		Paweł	  \textsc{Ogorzały} \\ [3cm]\

		
		%----------------------------------------------------------------------------------------
		%	DATE SECTION
		%----------------------------------------------------------------------------------------
		
		{\large \today}\\[3cm] % Date, change the \today to a set date if you want to be precise
		
		%----------------------------------------------------------------------------------------
		%	LOGO SECTION
		%----------------------------------------------------------------------------------------
		
		%\includegraphics{Logo}\\[1cm] % Include a department/university logo - this will require the graphicx package
		
		%----------------------------------------------------------------------------------------
		
		\vfill % Fill the rest of the page with whitespace
		
	\end{titlepage}
	
	%SPIS TRESI
	%
	%
	%
	%
	%
	%
	%
	%
	%
	
	\tableofcontents
	\vfill

	\section{Akt założycielski i statut spółki}
Pierwszym etapem założenia Spółki jest sporządzenie jej statutu. Statut ten powinien być sporządzony w formie aktu notarialnego. Statut musi zawierać::\\
firmę i siedzibę spółki,\\
Kraków Kapitol p.900\\
przedmiot jej działalności,\\
działalność wytwórcza i handlowa\\
\subsection{kapitał}
wysokość kapitału zakładowego oraz kwotę wpłaconą przed zarejestrowaniem na pokrycie kapitału zakładowego,\\
200000zł, 200000zł
\subsection{Akcje}
wartość nominalną akcji i ich liczbę ze wskazaniem, czy akcje są imienne, czy na okaziciela,
10zł za sztukę, 10000 sztuk
liczbę akcji poszczególnych rodzajów i związane z nimi uprawnienia, jeżeli mają być wprowadzone akcji różnych rodzajów,\\
emisja typu A – całość wyemitowanych akcji, TO MY\\
				
emisja typu B – emisja typu A pomniejszona o akcje znajdujące się poza rynkiem,\\
				
emisja typu C – emisja typu B pomniejszona o akcje posiadane przez dużych inwestorów.\\
\subsection{założyciele}
nazwiska i imiona albo firmy (nazwy) założycieli,\\
Katarzyna Kosiak, Paweł Ogorzały, Sebastian Katszer, Marcin Jędrzejczyk\\
\subsection{Organy spółki}
liczbę członków zarządu i rady nadzorczej albo co najmniej minimalną lub maksymalną liczbę członków tych organów oraz podmiot uprawniony do ustalenia składu zarządu lub rady nadzorczej, pismo do ogłoszeń, jeżeli Spółka zamierza dokonywać ogłoszeń również poza Monitorem Sądowym i Gospodarczym\\
Zarząd:Katarzyna Kosiak, Paweł Ogorzały, Sebastian Katszer, Marcin Jędrzejczyk\\
Rada nadzorcza:Katarzyna Kosiak, Paweł Ogorzały, Sebastian Katszer, Marcin Jędrzejczyk\\

\section{Postanowienia}
Statut powinien również zawierać, pod rygorem bezskuteczności wobec Spółki, postanowienia dotyczące:\\
liczby i rodzajów tytułów uczestnictwa w zysku lub w podziale majątku Spółki oraz związanych z nimi praw, wszelkich związanych z akcjami obowiązków świadczenia na rzecz Spółki, poza obowiązkiem wpłacenia należności za akcje,
warunków i sposobu umorzenia akcji,
ograniczeń zbywalności akcji,
co najmniej przybliżoną wielkość wszystkich kosztów poniesionych lub obciążających Spółkę w związku z jej utworzeniem, ustaloną na dzień związania
 Spółki,
uprawnień osobistych przyznawanych akcjonariuszom, o których mowa w art. 354.



\end{document}

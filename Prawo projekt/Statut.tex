\documentclass[a4paper, 11pt]{article}
\usepackage[polish]{babel}
\usepackage[MeX]{polski}
\usepackage[utf8]{inputenc}
\usepackage[T1]{fontenc}
%\usepackage{times}
\usepackage{graphicx,wrapfig}
%\usepackage{anysize}
%\usepackage{tikz}
%\usetikzlibrary{calc,through,backgrounds,positioning}
\usepackage{anysize}
\usepackage{float}
%\usepackage{stmaryrd}
%\usepackage{amssymb}
%\usepackage{amsthm}
%\marginsize{3cm}{3cm}{3cm}{3cm}
%\usepackage{amsmath}
%\usepackage{color}
%\usepackage{listings}
%\usepackage{enumerate}
%\lstloadlanguages{Ada,C++}


\begin{document}	
	% \noindent -  w tym akapicie nie bedzie wciecia
	% \ indent - to jest aut., ale powoduje ze jest wciecie
	% \begin{flushleft}, flushright, center - wyrownianie akapitu
	% \textbf{pogrubiany tekst}
	% \textit{kursywa} 
	% 					STRONY 
	%  http://www.codecogs.com/latex/eqneditor.php 
	%  http://www.matematyka.pl/latex.htm
	% 
	\begin{titlepage}
	
	
		
		\newcommand{\HRule}{\rule{\linewidth}{0.5mm}} % Defines a new command for the horizontal lines, change thickness here
		
		\center % Center everything on the page
		
		%----------------------------------------------------------------------------------------
		%	HEADING SECTIONS
		%----------------------------------------------------------------------------------------
		
		\textsc{\LARGE ZE WZORU NA STATUT SPÓŁKI AKCYJNEJ}\\[1.5cm] % Name of your university/college
		%\textsc{\Large Kraków}\\[0.5cm] % Major heading such as course name
		%\textsc{\large }\\[0.5cm] % Minor heading such as course title
		
		%----------------------------------------------------------------------------------------
		%	TITLE SECTION
		%----------------------------------------------------------------------------------------
		



		\HRule \\[0.4cm]
		{\fontsize{38}{50}\selectfont STATUT SPÓŁKI AKCYJNEJ }
	%	{ \Huge \bfseries} Symulator pożaru lasu\\[0.3cm] % Title of your document
		\HRule \\[1.5cm]
		
		%----------------------------------------------------------------------------------------
		%	AUTHOR SECTION
		%----------------------------------------------------------------------------------------
		
		% If you don't want a supervisor, uncomment the two lines below and remove the section above
		\Large \emph{Autorzy:}\\
		Marcin \textsc{Jędrzejczyk}\\ % Your name
		Sebastian \textsc{Katszer}\\ % Your name
		Katarzyna \textsc{Kosiak} \\ % Your name
		Paweł	  \textsc{Ogorzały} \\ [3cm]\

		
		%----------------------------------------------------------------------------------------
		%	DATE SECTION
		%----------------------------------------------------------------------------------------
		
		{\large \today}\\[3cm] % Date, change the \today to a set date if you want to be precise
		
		%----------------------------------------------------------------------------------------
		%	LOGO SECTION
		%----------------------------------------------------------------------------------------
		
		%\includegraphics{Logo}\\  % Include a department/university logo - this will require the graphicx package
		
		%----------------------------------------------------------------------------------------
		
		\vfill % Fill the rest of the page with whitespace
		
	\end{titlepage}
	
	%SPIS TRESI
	%
	%
	%
	%
	%
	%
	%
	%
	%
	
	\tableofcontents
	\vfill


\section{POSTANOWIENIA OGÓLNE} 

§ 1\\

Stawający zawiązują jako założyciele spółkę akcyjną, której firma brzmi: Platri zwaną dalej spółką.\\ 

§ 2\\

Siedzibą spółki jest  Kraków 30-732 ul. Wiślana 15.\\ 

§ 3\\

Przedmiotem działalności spółki jest działalność wytwórcza i handlowa.\\ 

§ 4\\

Spółka działa na obszarze Rzeczypospolitej Polskiej i za granicą. Spółka zakłada i prowadzi delegatury, oddziały w kraju i za granicą, a także uczestniczy w innych spółkach w kraju i za granicą.\\ 

§ 5 \\

Czas trwania spółki jest nieograniczony.\\ 

§ 6\\

\begin{enumerate}
\item	Kapitał akcyjny spółki wynosi 200 000 zł. (słownie dwieście tysięcy zł.) i dzieli się na 10000 (słownie dziesięć tysięcy ) akcji po  10  zł. (słownie dziesięć zł.) każda.
\item	Spółka ma prawo emitować akcje imienne i akcje na okaziciela.
\item	Wprowadza się akcje imienne serii A w liczbie 1000 o numeracji od 1 do 1000 na łączną wartość 10000zł (słownie sześćdziesiąt tysięcy zł.), które będą uprzywilejowane w następujący sposób:
\begin{itemize}
\item jedna akcja daje prawo do dwóch głosów,
\item akcje te mają pierwszeństwo w prawie podziału majątku likwidacyjnego spółki przed innymi akcjami.
\end{itemize}

\item	Akcje serii A obejmują: 1000 o numerach od 1 do 1000 na łączną kwotę 10000zł (słownie sześćdziesiąt tysięcy zł.) na nazwisko Katarzyna Kosiak 250 akcji od 1 do 250, Sebastian Katszer 250 akcji od 251 do 500, Paweł Ogorzały 250 akcji od 501 do 750, Marcin Jędrzejczyk 250 akcji od 751 do 1000.
\item	Wprowadza się akcje na okaziciela serii B w liczbie 9000 o numerach  1001-10000 na łączną kwotę 90000 zł. (słownie dziewięćdziesiąt tysięcy zł.). Akcje serii B objęte zostaną publiczną subskrypcją.
\item W terminie do dnia 2 stycznia 2016 roku cena emisyjna akcji serii B wyższa jest od ich wartości nominalnej o 10 procent.

\end{enumerate}

§ 7\\

\begin{enumerate}
\item	Kapitał akcyjny może być obniżony przez zmniejszenie nominalnej wartości akcji lub przez umorzenie części akcji.
\item	Spółka może podwyższać kapitał akcyjny w drodze emisji nowych akcji przez dotychczasowych akcjonariuszy. Decyzje na wniosek zarządu podejmuje rada nadzorcza w drodze uchwały.
\end{enumerate}
§ 8 \\

\begin{enumerate}
\item Akcje mogą być umarzane w przypadku, gdy:\\
a)	uchwalone zostanie obniżenie kapitału akcyjnego,\\
b)	spółka nabędzie akcje w drodze egzekucji swoich roszczeń.\\

\item	Spółce nie wolno na swój rachunek nabywać ani przyjmować w zastaw własnych akcji. Wyjątek stanowi nabycie w drodze egzekucji na zaspokojenie roszczeń spółki, których nie można zaspokoić z innego majątku akcjonariusza oraz nabycie w celu umorzenia.
\item	Jeśli akcje nabyte w drodze egzekucji nie zostaną zbyte w ciągu 1 roku od dnia nabycia, muszą być umorzone według przepisów o obniżenie kapitału akcyjnego.
\item	Umorzenie bez zachowania przepisów o obniżeniu kapitału akcyjnego może być dokonane jedynie z czystego zysku.
\end{enumerate}
§ 9\\

\begin{enumerate}
\item	Wszystkie akcje serii A i B mają takie samo prawo do dywidendy. Akcje dalszych serii nie korzystają z tego uprzywilejowania.
\item	Zysk bilansowy przeznaczony przez Walne zgromadzenie do podziału rozdziela się w stosunku do minimalnej wartości akcji.
\end{enumerate}
\section{WŁADZE SPÓŁKI}

Władzami spółki są: Zarząd, Rada nadzorcza i Walne zgromadzenie akcjonariuszy.\\

\subsection{ZARZĄD}

§ 10\\

\begin{enumerate}
\item	Członków zarządu wybiera i zwołuje w drodze uchwały walne zgromadzenie.
\item	Zarząd spółki składa się z 3 członków. Kadencja pierwszego zarządu trwa 2 lata, kadencja każdego następnego zarządu - 3 lata.
\item	Prezesa wybiera walne zgromadzenie spośród członków zarządu, Zawiera z nim umowę o pracę na stanowisku Dyrektora Generalnego. Podpisuje z nim umowę Rada Nadzorcza, która w stosunkach z zarządem reprezentuje spółkę.
\item	Mandaty członków zarządu wygasają z dniem odbycia Walnego zgromadzenia zatwierdzającego sprawozdanie, bilans, rachunek zysków i strat za ostatni rok urzędowania.
\item	Członko
\end{enumerate}
wie zarządu mogą być w każdej chwili odwołani przez Walne zgromadzenie, co nie uwłaszcza ich roszczeniom wynikającym z umowy o pracę.

§ 11\\

\begin{enumerate}
\item	Zarząd upoważniony jest do podejmowania wszelkich decyzji nie zastrzeżonych kompetencjami innych organów spółki.
Zarząd zobowiązany jest zarządzać majątkiem i sprawami spółki z należytą starannością wymaganą w obrocie gospodarczym, przestrzegać praw, postanowień niniejszego statutu oraz uchwał podjętych przez Walne zgromadzenie i radę nadzorczą w granicach ich kompetencji.
\item	Zarząd reprezentuje spółkę w sądzie i poza sądem.
\item	Zarząd działa w oparciu o niniejszy statut oraz Regulamin Pracy Zarządu uchwalony przez Walne Zgromadzenie.

\end{enumerate}

§ 12\\

\begin{enumerate}
\item	Do składania oświadczeń w zakresie praw i obowiązków majątkowych spółki oraz do podpisywania w imieniu spółki upoważnieni są:
\begin{enumerate}
\item	samodzielnie - Dyrektor Generalny spółki - do kwoty 150 000 zł. (słownie sto pięćdziesięciu tysięcy zł.)
\item	dwie osoby działające łącznie, spośród:\\
- pozostałych członków zarządu spółki,\\
- lub jeden z członków zarządu łącznie z prokurentem\\
\end{enumerate}

\item	Do wykonywania czynności określonego rodzaju lub specjalnych poruczeń mogą być ustanowieni pełnomocnicy, działający samodzielnie w granicach pisemnie udzielonego im przez zarząd pełnomocnictwa.
\item	Do ustanowionej prokury i pełnomocnictwa wymagana jest zgoda wszystkich członków zarządu. Prokura i pełnomocnictwo mogą być odwołane decyzją każdego z członków zarządu.
\end{enumerate}


§ 13\\
Pracownicy spółki podlegają rozporządzeniom zarządu. W szczególności zarząd przyjmuje i zwalnia pracowników oraz ustala należne wynagrodzenie w granicach etatu zatwierdzonego przez Radę nadzorczą.

\subsection{RADA NADZORCZA}

§ 14\\
\begin{enumerate}
\item	Radę Nadzorczą wybiera walne zgromadzenie.
\item	Rada Nadzorcza składa się z 4 członków, z czego 2 członków może być powołanych spoza grona akcjonariuszy.
\item	Walne zgromadzenie może w drodze uchwały podwyższyć ogólną liczbę członków Rady Nadzorczej, określając wówczas liczbę członków, których obecność na posiedzeniu jest konieczna dla ważności uchwał Rady Nadzorczej.
\item	Członkowie pierwszej Rady Nadzorczej powołani są na 1 rok. Członkowie następnych Rad Nadzorczych powoływani są na 3 lata. Mandaty członków Rady Nadzorczej wygasają z dniem odbycia Walnego zgromadzenia zatwierdzającego sprawozdanie, bilans, rachunek zysków i strat za ostatni rok ich urzędowania.
\item	Ustępujący członkowie Rady Nadzorczej mogą być ponownie wybrani.
\item	W przypadku ustąpienia członka Rady Nadzorczej w czasie trwania jej kadencji, skład Rady Nadzorczej może być uzupełniony do wymaganej liczby członków przez samą Radę Nadzorczą pod warunkiem zatwierdzenia zmian na najbliższym walnym zgromadzeniu.

\end{enumerate}

§ 15\\
\begin{enumerate}
\item	Rada Nadzorcza wybiera prezesa i wiceprezesa na swoim pierwszym posiedzeniu w tajnym głosowaniu, bezwzględną większością głosów obecnych na posiedzeniu Rady Nadzorczej.
\item	Członkowie Rady Nadzorczej wykonują swe prawa i obowiązki tylko osobiście i w sposób łączny.
\item	Posiedzenie Rady Nadzorczej zwołuje prezes, a w razie jego nieobecności wiceprezes. Posiedzenie Rady Nadzorczej odbywa się raz na dwa miesiące, a w miarę potrzeby częściej.
\item	Na wniosek zarządu posiedzenie Rady Nadzorczej powinno się odbyć w terminie najpóźniej 14 dni od daty zgłoszenia wniosku prezesowi lub wiceprezesowi.

\end{enumerate}

§ 16\\
\begin{enumerate}
\item	Uchwały Rady Nadzorczej zapadają bezwzględną większością głosów członków rady obecnych na posiedzeniu. W razie takiej samej ilości głosów decyduje głos przewodniczącego posiedzenia.
\item	Aby uchwały Rady Nadzorczej były ważne wymagane jest zaproszenie wszystkich członków Rady oraz obecność na posiedzeniu przynajmniej 4 członków.
\item	Członkowie Rady Nadzorczej otrzymują wynagrodzenie za pełnienie swoich obowiązków w wysokości średniego wynagrodzenia godzinowego w spółce z ostatnich trzech miesięcy za każdą godzinę pracy członka Rady.
\item	Rada Nadzorcza działa w oparciu o niniejszy Statut oraz Regulamin określający jej organizację i sposób wykonywania czynności, uchwalony przez walne zgromadzenie.
\end{enumerate}


§ 17\\

Rada Nadzorcza sprawuje stały nadzór nad działalnością spółki. Do zakresu działania Rady należy:
\begin{enumerate}
\item	rozpatrywanie rocznych programów działalności spółki oraz ich niezbędnych korekt,
\item	badanie bilansu oraz rachunków zysków i strat, zarówno co do zgodności z księgami i dokumentami, jak i stanem faktycznym, badanie sprawozdań okresowych i rocznych zarządu oraz wniosków zarządu co do podziału zysków i pokrycia strat oraz składanie walnemu zgromadzeniu pisemnego sprawozdania z wyników badań,
\item	zawieszenie z ważnych powodów w czynnościach poszczególnych lub wszystkich członków zarządu, jak również delegowanie członków Rady Nadzorczej do czasowego wykonywania czynności członków zarządu nie mogących sprawować swoich czynności,
\item	stawianie wniosków na walne zgromadzenie o udzielenie zarządowi zwolnienia z wykonania obowiązków,
\item	rozpatrywanie i opiniowanie wszystkich spraw mających być przedmiotem uchwał Walnego zgromadzenia,
\item	zawieranie, dokonywanie zmian i rozwiązywanie umów o pracę z członkami zarządu na podstawie treści uchwał Walnego zgromadzenia.

\end{enumerate}

\subsection{ WALNE ZGROMADZENIE}

§ 18\\
\begin{enumerate}
\item	Walne zgromadzenia mają być zwyczajne i nadzwyczajne.
\item	Zwyczajne walne zgromadzenie powinno odbyć się w terminie nie późniejszym niż 3 miesiące po upływie każdego roku obrotowego.
\item	Nadzwyczajne walne zgromadzenie zwołuje zarząd z własnej inicjatywy albo na żądanie akcjonariuszy, reprezentujących co najmniej 1/10 kapitału akcyjnego. Nadzwyczajne walne zgromadzenie powinno być zwołane w ciągu 14 dni od przedstawienia żądania wraz z projektami porządku obrad i wnioskowania uchwał.
\item	Walne zgromadzenie zwołuje się za pomocą listów poleconych, wysłanych co najmniej 14 dni przed wyznaczonym terminem. W zawiadomieniu powinien znajdować się przewidywany porządek obrad.
\end{enumerate}


§ 19\\
\begin{enumerate}
\item	Akcjonariusze uczestniczą w walnym zgromadzeniu osobiście lub przez ustanowionego na piśmie pełnomocnika.
\item	Akcjonariusze uczestniczący w walnym zgromadzeniu posiadają taką liczbę głosów jak liczbę akcji. Wyjątek stanowią akcje uprzywilejowanych serii A. 
\item	Walnemu zgromadzeniu przewodniczy jeden wybrany przez to zgromadzenie przedstawiciel akcjonariuszy. Przewodniczący powinien zostać wybrany przed przystąpieniem do obrad.
\item	Prezes Rady Nadzorczej otwiera posiedzenie walnego zgromadzenia, a w przypadku jego nieobecności wiceprezes, który przewodniczy do czasu wyboru przewodniczącego walnego zgromadzenia. Sprawy na walnym zgromadzeniu przedstawia Rada Nadzorcza.

\end{enumerate}

§ 20\\
\begin{enumerate}
\item	Walne zgromadzenie jest zdolne do powzięcia wiążących uchwał, o ile jest na nim reprezentowane 2/3 liczby akcji.
\item	Wszystkie uchwały walnego zgromadzenia zapadają większością 2/3 głosów oddanych, chyba że przepisy Kodeksu Handlowego ustanawiają surowsze warunki powzięcia tych uchwał.
\end{enumerate}


§ 21\\

Uchwały walnego zgromadzenia w sprawie przedmiotu przedsiębiorstwa spółki nie wymagają dla swej ważności wykupu akcji tych akcjonariuszy, którzy nie zgadzają się na zmianę.\\ 

§ 22\\

Do kompetencji walnego zgromadzenia należy;
\begin{enumerate}
\item ustanowienie na wniosek zarządu kierunków rozwoju spółki oraz programów jej działalności,
\item	rozpatrywanie i zatwierdzanie sprawozdań zarządu, bilansu oraz rachunków zysków i strat za ubiegły rok obrotowy,
\item	podział zysków i pokrycie strat oraz przeznaczenie funduszu rozwoju i funduszów rezerwowych,
\item	udzielenie radzie nadzorczej i zarządowi skwitowania z wykonania obowiązków,
\item	wybór i odwoływanie członków rady nadzorczej oraz zarządu,
\item	powiększenie lub zmniejszenie kapitału zakładowego,
\item	zmiana statutu spółki nie wyłączając zmiany przedmiotu działalności,
\item	udzielenie zezwolenia na powoływanie przez zarząd oddziałów spółki, uczestniczenie w innych spółkach oraz łączenie spółek,
\item	rozwiązanie i likwidacja spółki,
\item	rozpatrywanie i rozstrzyganie wniosków przedstawionych przez radę nadzorczą,
\item	uchwalenie regulaminu obrad walnego zgromadzenia, regulaminu pracy rady nadzorczej oraz regulaminu pracy zarządu spółki,
\item	inne sprawy przewidziane przepisami Kodeksu Handlowego.
\end{enumerate}


\section{RACHUNKOWOŚĆ SPÓŁKI}
 
	 § 23\\

\begin{enumerate}
\item	Rokiem obrotowym spółki jest rok kalendarzowy.
\item	Pierwszy rok obrachunkowy kończy się 31 grudnia 2016r.
\end{enumerate}


§ 24\\

Zarząd spółki obowiązany jest w ciągu 2 miesięcy po upływie roku obrotowego sporządzić i złożyć radzie nadzorczej bilans na ostatni dzień roku obrotowego, rachunek zysków i strat za rok ubiegły oraz dokładne pisemne sprawozdanie z działalności spółki w tym okresie.\\ 

§ 25\\

Spółka tworzy następujące fundusze w ciężar zysku:
\begin{enumerate}
\item	Fundusz zapasowy, na który przekazywane jest co najmniej 8% zysku do podziału do momentu, gdy fundusz ten osiągnie przynajmniej 1/3 kapitału akcyjnego,
\item	Fundusz rozwoju, na który przekazywane są środki w wysokości zatwierdzonej w ustawie walnego zgroma-dzenia.
\item	Fundusz dla załogi tworzony z zysku po opodatkowaniu, rozdzielany jest w wysokości i zgodnie z regulaminami opracowanymi przez zarząd spółki i zatwierdzonymi przez radę nadzorczą, o ile walne zgromadzenie podejmie uchwałę o utworzeniu takiego funduszu.

\end{enumerate}

§ 26\\

Spółka tworzy fundusze obciążające koszty działalności:
\begin{enumerate}
\item	Fundusz mieszkaniowy
\item	Fundusz socjalny
\item	inne fundusze określone obowiązującymi przepisami
\end{enumerate}


\section{PRZEPISY KOŃCOWE}
 
	 § 27\\

W razie utraty kapitału zapasowego i jednej trzeciej kapitału zakładowego walne zgromadzenie może uchwalić rozwiązanie spółki. Można również postawić wniosek o rozwiązanie spółki w przypadku, gdy po utraceniu kapitału zapasowego interesy spółki nie dostarczyły co najmniej przez dwa kolejne lata żadnej dywidendy akcjonariuszom. Z takim wnioskiem mogą wystąpić akcjonariusze dysponujący jedną trzecią kapitału akcyjnego spółki.\\ 

§ 28\\

Rozwiązanie spółki następuje po przeprowadzeniu likwidacji. Likwidację prowadzi się pod firmą spółki z dodatkiem "w likwidacji".\\
Musi być dwóch likwidatorów, a będą nimi osoby, które zostały wybrane przez walne zgromadzenie. Likwidatorzy firmę spółki będą podpisywać łącznie.\\ 

§ 29\\

W sprawach nie uregulowanych niniejszym statutem mają zastosowanie przepisy Kodeksu Handlowego i innych obowiązujących aktów prawnych.\\ 

§ 30\\

Spółka będzie zamieszczać swoje ogłoszenia w gazecie codziennej “Parkiet Gazeta Giełdy“.\\ 

§ 31\\

Wszelkie spory, jakie powstaną na tle niniejszego statutu rozpatrywane będą przez sąd właściwy ze względu na siedzibę spółki.\\ 

§ 32\\

Koszta tego aktu ponosi spółka. Wpisy należy wydawać wspólnikom i spółce w dowolnej ilości.\\ 

§ 33\\

Pobrano:\\
a)	opłatę skarbową z §58 rozp. o opłatach skarbowych; \\
b)	opłatę notarialną z §3 i 9 o opłatach notarialnych.\\
Ogółem 2570 zł.\\
Akt ten odczytano, przyjęto i podpisano.\\

Podpisy stron, poświadczenie o pobranych opłatach\\
znaczkami oraz podpis notariusza\\
..........................................\\
..........................................\\
..........................................\\


\end{document}